\chapter{\label{app:sums}Spectator Sums}

In the derivation of a two-body effective potential for the 3N interaction, we encounter sums over the spin, isospin, and momentum quantum numbers for the Fermi gas of spectator particles. In this appendix, we show how to perform these sums.

\section{Spin Sums}
Recall that the Fermi gas is assumed to be spin symmetric. The spin operators occur either in the form $\sigma_1^i\sigma_2^j$ or $\sigma_1^i\sigma_2^j\sigma_3^k$. We begin with the two-operator product. Altogether there are thirty-six antisymmetrized diagrams. Of these, permutation of $\alpha_1\leftrightarrow \alpha_2$ and $\beta_1\leftrightarrow \beta_2$ reduces the number of unique calculations by a factor of four. Using Cartesian indices, we need to perform five unique sums over Pauli matrix products
\begin{align}
&\sum_{m_{s_\gamma}=\pm1/2} \bra{\alpha_1,\alpha_2, \gamma } \sigma_1^i\sigma_2^j \ket{\beta_1, \beta_2, \gamma} \label{eq:2sigma33}\\
&\sum_{m_{s_\gamma}=\pm1/2} \bra{\alpha_1,\alpha_2, \gamma } \sigma_1^i\sigma_2^j \ket{\beta_1,  \gamma, \beta_2 } \label{eq:2sigma32}\\
 %&\sum_{m_{s_\gamma}=\pm1/2} \bra{\alpha_1, \gamma, \alpha_2 } \sigma_1^i\sigma_2^j \ket{\beta_1, \beta_2, \gamma } \\
 &\sum_{m_{s_\gamma}=\pm1/2} \bra{\alpha_1, \gamma, \alpha_2 } \sigma_1^i\sigma_2^j \ket{\gamma, \beta_1, \beta_2 } \label{eq:2sigma21}\\
 &\sum_{m_{s_\gamma}=\pm1/2} \bra{\alpha_1, \gamma, \alpha_2 } \sigma_1^i\sigma_2^j \ket{\beta_1, \gamma, \beta_2 }. \label{eq:2sigma22}
\end{align}
All other permutations of initial and final indices may be found by using permutations of the operator indices, 
\begin{equation}
\sum_{m_{s_\gamma}=\pm1/2} \bra{ \gamma, \alpha_1, \alpha_2 } \sigma_1^i\sigma_2^j \ket{\beta_1, \gamma, \beta_2 } = \sum_{m_{s_\gamma}=\pm1/2} \bra{\alpha_1, \gamma, \alpha_2 } \sigma_1^j\sigma_2^i \ket{\gamma, \beta_1, \beta_2 },
\end{equation}
which generates four  of the remaining five summations; and by requiring hermiticity,
\begin{equation}
\sum_{m_{s_\gamma}=\pm1/2} \bra{\alpha_1, \gamma,\alpha_2, } \sigma_1^i\sigma_2^j \ket{\beta_1,  \beta_2, \gamma }=
\bigg(\sum_{m_{s_\gamma}=\pm1/2} \bra{\beta_1, \beta_2, \gamma } \sigma_1^i\sigma_2^j \ket{\alpha_1, \gamma, \alpha_2 }\bigg)^*
\end{equation}
which generates the final unique term.

The diagonal terms are the simplest. When the spin operator does not act on the Fermi gas state the summation produces a factor of two,
\begin{equation}\label{eq:2sigma33summed}
\sum_{m_{s_\gamma}=\pm1/2} \bra{\alpha_1,\alpha_2, \gamma } \sigma_1^i\sigma_2^j \ket{\beta_1, \beta_2, \gamma} = \bra{\alpha_1,\alpha_2 } 2 \sigma_1^i\sigma_2^j \ket{\beta_1, \beta_2}.
\end{equation}
Because the Pauli matrices are traceless, $\sum_\gamma\braket{\gamma| \sigma^i |\gamma}=0$ which implies that the sum \eqref{eq:2sigma22} vanishes,
\begin{equation}\label{eq:2sigma22summed}
\sum_{m_{s_\gamma}=\pm1/2} \bra{\alpha_1, \gamma, \alpha_2 } \sigma_1^i\sigma_2^j \ket{\beta_1, \gamma, \beta_2 }=0.
\end{equation}

In evaluating \eqref{eq:2sigma32} we see that, since no spin operator acts on particle three, we generate a Kronecker delta which selects one term out of the sum,
\begin{equation}
\begin{split}
\sum_{m_{s_\gamma}=\pm1/2} \bra{\alpha_1,\alpha_2, \gamma } \sigma_1^i\sigma_2^j \ket{\beta_1,  \gamma, \beta_2 }
& = \sum_{m_{s_\gamma}=\pm1/2}  \delta_{m_{s_\gamma},m_{s_{\beta_1}}} \bra{\alpha_1,\alpha_2 } \sigma_1^i\sigma_2^j \ket{\beta_1,  \gamma } \\
& =  \bra{\alpha_1,\alpha_2 } \sigma_1^i\sigma_2^j \ket{\beta_1,  \beta_2 }.
\end{split}
\end{equation}

The final term  \eqref{eq:2sigma21} will generate a two body interaction with spin operators acting only on a single particle. 
\begin{equation}\begin{split}
\sum_{m_{s_\gamma}=\pm1/2} \bra{\alpha_1, \gamma, \alpha_2 } \sigma_1^i\sigma_2^j \ket{\gamma, \beta_1, \beta_2 }
&= \sum_{m_{s_\gamma}=\pm1/2} \bra{\alpha_1} \sigma^i \ket{\gamma}\bra{\gamma}\sigma^j\ket{\beta_1}\braket{\alpha_2 | \beta_2} \\
&=\bra{\alpha_1, \alpha_2} \sigma_1^i \sigma_1^j\ket{\beta_1,\beta_2} \\
&=\bra{\alpha_1, \alpha_2} \delta^{ij}+i \epsilon^{ijk}\sigma_1^k \ket{\beta_1,\beta_2}
\end{split}
\end{equation}

We also evaluate the analogs of these sums for case of three spin operators which arises in the $V_4$ term. We can reduce our evaluation to two forms, as the spin operator structure is invariant under permutations of any two particles. First, the traceless property implies that all terms with diagonal spectator quantum numbers are zero, 
\begin{align}\label{eq:3sigma33summed}
&\sum_{m_{s_\gamma}=\pm1/2} \bra{\alpha_1,\alpha_2, \gamma } \sigma_1^i\sigma_2^j\sigma_3^k \ket{\beta_1, \beta_2, \gamma} =0
\end{align}
in analogy to \eqref{eq:2sigma22summed}. The only remaining unique term is
\begin{equation}\begin{split}\label{eq:3sigma32summed}
\sum_{m_{s_\gamma}=\pm1/2} \bra{\alpha_1, \alpha_2,  \gamma } \sigma_1^i\sigma_2^j\sigma_3^k \ket{ \beta_1, \gamma, \beta_2 }
&= \sum_{m_{s_\gamma}=\pm1/2} \bra{\alpha_1} \sigma^i \ket{\beta_1}\bra{\alpha_2}\sigma^j\ket{\gamma}\bra{\gamma}\sigma^k\ket{ \beta_2} \\
&=\bra{\alpha_1, \alpha_2} \sigma_1^i \sigma_2^j\sigma_2^k \ket{\beta_1,\beta_2} \\
&=\bra{\alpha_1, \alpha_2} \sigma_1^i(\delta^{jk}+i \epsilon^{jkl}\sigma_2^l) \ket{\beta_1,\beta_2}
\end{split}
\end{equation}
This arises only for one diagram in the two-pion exchange. Using the results of \eqref{eq:3sigma32summed} and standard identities for the permutation symbol $\epsilon$, we can simplify the resulting expression further,
\begin{multline}
\sum_{m_{s_\gamma}=\pm1/2} \bra{\alpha_1, \alpha_2,  \gamma }\vec{\sigma}_1\cdot \vec{q}_1\,\vec{\sigma}_2\cdot \vec{q}_2 \,\vec{\sigma}_3\cdot(\vec{q}_1\times\vec{q}_2) \ket{ \beta_1, \gamma, \beta_2 } = \\
 \bra{\alpha_1, \alpha_2 }i\Big[\vec{\sigma}_1\cdot \vec{q}_1\,\vec{\sigma}_2\cdot \vec{q}_1 \,(\vec{k}'-\vec{k}_\gamma)^2-\vec{\sigma}_1\cdot \vec{q}_1\,\vec{\sigma}_2\cdot (\vec{k}'-\vec{k}_\gamma) \,\vec{q}_1\cdot(\vec{k}'-\vec{k}_\gamma) \Big] \ket{ \beta_1, \beta_2 } .
\end{multline}
Once coupled to the momentum operators, we also find a somewhat different looking expression for
\begin{multline}
\sum_{m_{s_\gamma}=\pm1/2} \bra{\gamma, \alpha_1, \alpha_2 }\vec{\sigma}_1\cdot \vec{q}_1\,\vec{\sigma}_2\cdot \vec{q}_2 \,\vec{\sigma}_3\cdot(\vec{q}_1\times\vec{q}_2) \ket{ \beta_1, \gamma, \beta_2 } = \\
 \bra{\alpha_1, \alpha_2 } \Big(
\vec{\sigma}_2\cdot \left[(\vec{k}_\gamma-\vec{k})\times(\vec{k}'-\vec{k}_\gamma)\right] \,(\vec{k}_\gamma-\vec{k})\cdot(\vec{k}'-\vec{k}_\gamma) \hspace{2cm}\\
-i \vec{\sigma}_1\cdot \left[(\vec{k}_\gamma-\vec{k})\times(\vec{k}'-\vec{k}_\gamma)\right] \vec{\sigma}_2\cdot \left[(\vec{k}_\gamma-\vec{k})\times(\vec{k}'-\vec{k}_\gamma)\right]
\Big) \ket{ \beta_1, \beta_2 }.
\end{multline}

\section{Isospin Projections}
We allow the Fermi gas to be asymmetric in isospin via the inclusion of different momentum-dependent density of states. In practice, this means that we project different linear combinations of the momentum integrals described in Section~\ref{app:momSums} according to \eqref{eq:fullSum}. These projections are dependent on the two-body isospin quantum numbers, and we derive these dependences here by expanding the spectator state as 
\begin{equation}
\ket{\gamma}=\delta_{m_{\tau_\gamma},\nicefrac{1}{2}}\ket{\tfrac{1}{2}}+\delta_{m_{\tau_\gamma},\nicefrac{-1}{2}}\ket{-\tfrac{1}{2}}.
\end{equation}
This implies that we can write the outer product in terms of projection operators as,
\begin{equation}\label{eq:tauProjections}
\ket{\gamma}\bra{\gamma} = \delta_{m_{\tau_\gamma},\nicefrac{1}{2}}\dfrac{1+\tau^3}{2}+\delta_{m_{\tau_\gamma},\nicefrac{-1}{2}}\frac{1-\tau^3}{2}.
\end{equation}

Compared to the spin sums, the isospin operators are never coupled to other operators and in fact take only two forms, $\vec{\tau}_1\cdot\vec{\tau}_2$ and $\vec{\tau_3}\cdot[\vec{\tau}_1\times\vec{\tau}_2]$. For the first form, the case where the spectator momentum is diagonal in the third particle mirrors \eqref{eq:2sigma33summed}:
\begin{equation}\begin{split}
\bra{\alpha_1,\alpha_2, \gamma } \vec{\tau}_1\cdot \vec{\tau}_2 \ket{\beta_1, \beta_2, \gamma} 
&= \bra{\alpha_1,\alpha_2 } \vec{\tau}_1\cdot \vec{\tau}_2 \ket{\beta_1, \beta_2} \braket{\gamma|\gamma} \\
&= \bra{\alpha_1,\alpha_2 } \vec{\tau}_1\cdot \vec{\tau}_2 \ket{\beta_1, \beta_2} \left(\delta_{m_{\tau_\gamma},\nicefrac{1}{2}}+\delta_{m_{\tau_\gamma},\nicefrac{-1}{2}} \right) 
\end{split}
\end{equation}
The other diagonal terms do not contribute due to the spin sums \eqref{eq:2sigma22summed} and \label{eq:3sigma32summed} and need not be calculated. For the remaining terms, we make use of \eqref{eq:tauProjections}. First,
\begin{equation}\begin{split}
\bra{\alpha_1, \alpha_2, \gamma } \vec{\tau}_1\cdot \vec{\tau}_2 \ket{\beta_1,\gamma,  \beta_2} 
&= \bra{\alpha_1}\tau^i\ket{\beta_1}\braket{\alpha_2 | \tau^i | \gamma} \braket{\gamma | \beta_2} \\
&= \bra{\alpha_1}\tau^i\ket{\beta_1}\braket{\alpha_2 | \tau^i  \left(\delta_{m_{\tau_\gamma},\nicefrac{1}{2}}\dfrac{1+\tau^3}{2}+\delta_{m_{\tau_\gamma},\nicefrac{-1}{2}}\frac{1-\tau^3}{2}\right) | \beta_2} \\
&=\frac{1}{2} \bra{\alpha_1}\tau^i\ket{\beta_1}\bra{\alpha_2} \tau^i  \Delta_+ +\tau^i \tau^3 \Delta_-\ket{\beta_2}\\
&=\bra{\alpha_1,\alpha_2} \frac{\taudot\Delta_+ + \big(\tau_1^3-i\taucrossthree\big)\Delta_-}{2}\ket{\beta_1,\beta_2}
\end{split}
\end{equation}
where we define the isovector and isocalar projections $\Delta_\pm=\left(\delta_{m_{\tau_\gamma},\nicefrac{1}{2}}\pm\delta_{m_{\tau_\gamma},\nicefrac{-1}{2}} \right)$ for convenience. The other remaining projection of $\taudot$ is
\begin{equation}\begin{split}
\bra{\gamma, \alpha_1, \alpha_2 } \vec{\tau}_1\cdot \vec{\tau}_2 \ket{\beta_1,\gamma,  \beta_2} 
&= \bra{\alpha_1}\tau^i\ket{\gamma}\bra{\gamma}\tau^i\ket{\beta_1}\braket{\alpha_2|\beta_2} \\
&= \bra{\alpha_1}\tau^i\left(\frac{\Delta_+ + \tau^3\Delta_-}{2}\right)\tau^i\ket{\beta_1}\braket{\alpha_2|\beta_2} \\
&= \bra{\alpha_1,\alpha_2}\frac{3\Delta_+-\tau_1^3\Delta_-}{2}\ket{\beta_1,\beta_2}.
\end{split}
\end{equation}
For the $V_4$ term with the three coupled isospin operators, we know that the diagonal terms all vanish from the spin sums. The other terms are easy to evaluate from particle permutations of one example, which involves some tedious algebra to obtain:
\begin{equation}
\bra{ \alpha_1, \alpha_2, \gamma }\epsilon^{ijk}\tau_1^i\tau_2^j\tau_3^k \ket{\beta_1, \gamma, \beta_2}  
= \bra{\alpha_1 \alpha_2} i \left(\taudot\Delta_+-\tau_1^3\Delta_- \right) \ket{\beta_1 \beta_2}
\end{equation}


\section{\label{app:momSums}Momentum Space}

In this section we give explicit expressions for the sum over spectator momentum $\vec{k}_\gamma$ in the momentum space expressions of Section~\ref{sec:momentum} assuming the zero-temperature Fermi-Dirac density of states. For terms where only a single pion's momentum transfer depends on $\vec{k}_\gamma$ we have three distinct integrals of the form 
\begin{equation}
\Gamma_{\alpha}(k) = \frac{k^{2-\alpha}}{m_\pi^3}\int^{k_\gamma<k_F}\frac{d^3\vec{k}_\gamma}{(2\pi)^3}  \frac{\left\{1,\vec{k}_\gamma\cdot\hat{k},k_\gamma^2\right\}_\alpha}{(\vec{k}-\vec{k}_\gamma)^2+m_\pi^2}
\end{equation}
where $\left\{1,k_\gamma\cos\theta,k_\gamma^2\right\}_\alpha$ indicates that the numerator is $1$ for $\alpha=0$, $k_\gamma\cos\theta$ when $\alpha=1$, etc. We have suppressed the $n,p$ index on the Fermi momenta in this appendix, but remind the reader that these integrals appear in linear combinations of the isospin components.

The results of analytically integrating over the Fermi sphere are 
 \begin{align}
 \begin{split}
 \Gamma_{0}(k)&=\frac{k^2}{m_\pi^3}\int^{k_\gamma<k_F}\frac{d^3\vec{k}_\gamma}{(2\pi)^3} \frac{1}{(\vec{k}-\vec{k}_\gamma)^2+m_\pi^2} \\
 &=\frac{3}{4}\frac{\rho}{m_\pi^3}\left[  \tilde{k}^2 -\tilde{m}_\pi\tilde{k}^2\left(\arctan\frac{1-\tilde{k}}{\tilde{m}_\pi}+\arctan\frac{1+\tilde{k}}{\tilde{m}_\pi}\right)\right. \\
 &\left.\qquad\qquad\qquad\qquad\qquad+\frac{1}{4}\left(1-\tilde{k}^2+\tilde{m}_\pi^2\right)\log\frac{\tilde{m}_\pi^2+(\tilde{k}+1)^2}{\tilde{m}_\pi^2+(\tilde{k}-1)^2} \right],
 \end{split} \\
 %
  \begin{split}
 \Gamma_{1}(k)&=\frac{k}{m_\pi^3}\int^{k_\gamma<k_F}\frac{d^3\vec{k}_\gamma}{(2\pi)^3} \frac{\vec{k}_\gamma\cdot\hat{k}}{(\vec{k}-\vec{k}_\gamma)^2+m_\pi^2} \\
 &=\frac{3}{4}\frac{\rho}{m_\pi^3}\left[  (3\tilde{k}^2-\tilde{m}_\pi^2-1)-4\tilde{k}^2\tilde{m}_\pi\left(\arctan\frac{1-\tilde{k}}{\tilde{m}_\pi}+\arctan\frac{1+\tilde{k}}{\tilde{m}_\pi}\right)\right. \\
 &\left.\qquad\qquad\qquad\qquad\qquad+\frac{(1+\tilde{m}_\pi^2)^2-3\tilde{k}^4+2\tilde{k}^2(1+3\tilde{m}_\pi^2)}{4\tilde{k}}\log\frac{\tilde{m}_\pi^2+(\tilde{k}+1)^2}{\tilde{m}_\pi^2+(\tilde{k}-1)^2} \right],
 \end{split} \\
 %
  \begin{split}
 \Gamma_{2}(k)&=\frac{k}{m_\pi^3}\int^{k_\gamma<k_F}\frac{d^3\vec{k}_\gamma}{(2\pi)^3} \frac{k_\gamma^2}{(\vec{k}-\vec{k}_\gamma)^2+m_\pi^2} \\
 &=\frac{1}{2}\frac{\rho}{m_\pi^3}\left[  (3\tilde{k}^2-9\tilde{m}_\pi^2-1)-6\tilde{m}_\pi(\tilde{k}^2-\tilde{m}_\pi^2)\left(\arctan\frac{1-\tilde{k}}{\tilde{m}_\pi}+\arctan\frac{1+\tilde{k}}{\tilde{m}_\pi}\right)\right. \\
 &\left.\qquad\qquad\qquad\qquad\qquad+\frac{3\left(\tilde{k}^4+\tilde{m}_\pi^4-6\tilde{k}^2\tilde{m}_\pi^2-1\right)}{4\tilde{k}}\log\frac{\tilde{m}_\pi^2+(\tilde{k}+1)^2}{\tilde{m}_\pi^2+(\tilde{k}-1)^2} \right].
 \end{split} 
  \end{align}
All variables appear in the dimensionless combinations $\tilde{m}_\pi=m_\pi/k_F^{n,p}$ and $\tilde{k}=k/k_F^{n,p}$. For actual calculations, one can replace $k_F$ with the observable densities using \eqref{eq:kF}.

\section{\label{app:rhohat}Coordinate Space}

Dealing with the Fermi gas momentum is much simpler in coordinate space. After Fourier transforming the summed diagrams back into a configuration space potential, in all cases the dependence on the spectator momentum appears through the integral
\begin{equation}
\hat{\rho}(|\rot-\rotp|) = \frac{1}{m_\pi^3}\int \frac{d^3k_\gamma}{(2\pi)^3}n(k_\gamma)e^{i\vec{k}_\gamma \cdot(\rot-\rotp)}
\end{equation}
Using our standard choice $n(k_\gamma)=\theta(k_\gamma-k_F)$ this integral can be performed analytically,
\begin{equation}\begin{split}
\hat{\rho}(r) &= \frac{1}{(2\pi)^2}\frac{1}{m_\pi^3}\int_{-1}^{-1}du \int_0^{k_F} k_\gamma^2dk_\gamma \;e^{i \vec{k}_\gamma r u} \\
&= \frac{1}{(2\pi)^2}\frac{1}{m_\pi^3} \int_0^{k_F} k_\gamma^2 dk_\gamma \;\frac{e^{i\vec{k}_\gamma r}-e^{-i\vec{k}_\gamma r}}{ik_\gamma r} \\
&= \frac{1}{(2\pi)^2}\frac{1}{m_\pi^3} \int_0^{k_F} k_\gamma^2 dk_\gamma \;\frac{2 \sin(k_\gamma r)}{k_\gamma r} \\
&= \frac{k_F^2}{2\pi^2}\frac{1}{m_\pi^3} \frac{j_1(k_\gamma r)}{r}.
\end{split}
\end{equation}
We can replace the Fermi momentum with the density using \eqref{eq:kF} to obtain
\begin{equation}
\hat{\rho}(r) = \frac{1}{2}\left(\frac{3\rho}{\pi^2 m_\pi^3}\right)^{2/3} \frac{j_1( [3\pi^2\rho]^{1/3} r)}{m_\pi r}.
\end{equation}
Note that the factor of $1/2$ is not included in \eqref{eq:hatdensities} which means that
\begin{equation}
\int \frac{d^3k_\gamma}{(2\pi)^3}\big[ n(k_\gamma)\pm n(k_\gamma)\big]e^{i\vec{k}_\gamma \cdot(\rot-\rotp)} =m_\pi^3 \frac{ \rhohat{0,1}{|\rot-\rotp)|}}{2}.
\end{equation}

\chapter{\label{app:wNotation} Fourier Integrals}

In an attempt at organizing the terms in the coordinate space effective potential, we introduce the following notation for the dimensionless radial functions $\w{\alpha}{\beta}{\ell}{ m r }$:

\begin{align}
i^\ell \, Y_\ell(\hat{r}) \, \w{\alpha}{\beta}{\ell}{ m r}  &= \frac{4 \pi}{m^{3+\alpha-2\beta} }  \int \frac{d^3 q}{(2\pi)^3} \frac{q^\alpha}{(q^2+m^2)^\beta} Y_\ell(\hat{q}) \, e^{i \vec{q} \cdot \vec{r}  },
\end{align}
which implies that
\begin{align}\label{eq:wDef}
\w{\alpha}{\beta}{\ell}{z} =  \frac{2}{\pi} \int dk \, k^2 \, \frac{k^\alpha}{(1+k^2)^\beta} j_\ell(k z)
\end{align}
with $\vec{z} \equiv m \vec{r}$ and $k$ both dimensionless. In Table~\ref{table:wTable} we give the integrated results for selected values of $\alpha, \beta, \ell$. The first function we recover, for the case of $(\alpha, \beta, \ell ) = (0,1,0)$, is nothing but a Yukawa potential.

\begin{table}
\begin{center}
\renewcommand*{\arraystretch}{1.5}
\begin{tabular}{| c c c | c |}
\hline
$\alpha$ & $\beta$ & $\ell$ & $ \w{\alpha}{\beta}{\ell}{z}$ \\
\hline
0 & 0 & 0 & $ 4\pi \delta^{(3)}(\vec{z}) $ \\
0 & 1 & 0 & $ \frac{\displaystyle e^{-z}}{\displaystyle z} $ \\
2 & 1 & 0 & $ -\frac{\displaystyle e^{-z}}{\displaystyle z} + 4\pi \delta^{(3)}(\vec{z}) $ \\
1 & 1 & 1 &  $\frac{\displaystyle e^{-z}}{\displaystyle z} (1 + \tfrac{1}{z})  $ \\
2 & 1 & 2 & $ \frac{\displaystyle e^{-z}}{\displaystyle z} (1+ \tfrac{3}{z} + \tfrac{3}{z^2} )$ \\
0 & 2 & 0 & $ \frac{\displaystyle e^{-z}}{2} $ \\
2 & 2 & 0 & $ \frac{\displaystyle e^{-z}}{\displaystyle z} (-\tfrac{z}{2} + 1 ) $ \\
2 & 2 & 2 & $  \frac{\displaystyle e^{-z}}{\displaystyle z} (\tfrac{z}{2} + \tfrac{1}{2} ) $ \\
4 & 2 & 0 & $  \frac{\displaystyle e^{-z}}{\displaystyle z} (\tfrac{z}{2} - 2 ) + 4 \pi \delta^3( \vec{z} )  $ \\
4 & 2 & 2 & $  \frac{\displaystyle e^{-z}}{\displaystyle z} ( - \tfrac{z}{2} + \tfrac{1}{2} + \tfrac{3}{z} + \tfrac{3}{z^2} ) $ \\
\hline
\end{tabular}
\end{center}
\caption[Selected expressions for the scalar functions $\w{\alpha}{\beta}{l}{z}$]{\label{table:wTable}Table of selected values for $\w{\alpha}{\beta}{\ell}{z}$ found by integrating equation \eqref{eq:wDef}.}
\end{table}

To further demonstrate the use of this integral, we now derive the familiar tensor force resulting from one pion exchange. This amplitude is related to the integral
\begin{equation} 
V_\pi( \vec{r}) =  -\frac{g_A^2 }{4 F_\pi^2} \:\taudot \int \frac{d^3 q } {(2 \pi)^3 } \frac{(\vec{\sigma}_1 \cdot \vec{q} ) (\vec{\sigma}_2 \cdot \vec{q} ) } {q^2+m_\pi^2} e^{i \vec{q} \cdot \vec{r} }. 
\end{equation}
We can rewrite the numerator in the integral as
\begin{equation} (\vec{\sigma}_1 \cdot \vec{q}) ( \vec{\sigma}_2 \cdot \vec{q} )= q^2 \left(\sqrt{\frac{8\pi}{15}} [\vec{\sigma}_1 \otimes \vec{\sigma}_2]_2 \cdot Y_2(\hat{q}) + \tfrac{1}{3} \vec{\sigma}_1 \cdot \vec{\sigma}_2  Y_0(\hat{q}) \right),\end{equation}
which then gives
\begin{equation} 
V_\pi( \vec{r}) = - \frac{g_A^2 }{4 F_\pi^2} \taudot \left(\frac{m_\pi^3}{4 \pi}\right) \left\{ - \sqrt{\frac{8\pi}{15}} [\vec{\sigma}_1 \otimes \vec{\sigma}_2]_2 \cdot Y_2(\hat{r}) \w{2}{1}{2}{m_\pi r} + \frac{1}{3} \sigma_1 \cdot \sigma_2 \w{2}{1}{0}{m_\pi r } \right\},  \end{equation}
or, plugging in explicit values for the $w$'s from Table \ref{table:wTable} and recalling that the tensor operator $S_{12}$ is conventionally defined
\begin{equation}
S_{12} \equiv   3\vec{\sigma}_1\cdot \hat{r}\vec{\sigma}_2\cdot \hat{r} -\sigmadot = 3\:\sqrt{\frac{8\pi}{15}} \:[\sigma_1 \otimes \sigma_2]_2 \cdot Y_2(\hat{r}) , 
\end{equation}
we have
\begin{equation} V_\pi( \vec{r})= \frac{g_A^2 }{4 F_\pi^2}  \left(\frac{m_\pi^3}{12 \pi}\right) \left\{ \frac{e^{-m_\pi r}}{m_\pi r} \left[ S_{12}\left(1+\frac{3}{m_\pi r}+ \frac{3}{m_\pi^2 r^2} \right) +  \sigma_1 \cdot \sigma_2 \right] - 4\pi  \sigma_1 \cdot \sigma_2 \delta^3(m_\pi \vec{r}) \right\}, 
\end{equation}
which is the familiar result.

Representations in terms of these functions are not unique. When $\alpha=2\beta$, one may rewrite the numerator in \eqref{eq:wDef} using only powers of $q$ less than $\alpha$. As a concrete example, consider the case for $\w{2}{1}{0}{z}$. Because
\begin{equation}
\frac{q^2}{q^2+m^2}=1-\frac{m^2}{q^2+m^2}
\end{equation}
we immediately see that $\w{2}{1}{0}{z}=\w{0}{0}{0}{z}-\w{0}{1}{0}{z}$, a result which can be confirmed by inspection from Table~\ref{table:wTable}. This rewriting also makes clear the origin of the short-range delta function terms in the Fourier transforms of the form $\w{2\beta}{\beta}{0}{z}$. Other relationships include,
\begin{align}
\w{4}{2}{0}{z}&=\w{0}{0}{0}{z}-2 \w{2}{2}{0}{z}-\w{0}{2}{0}{z} \\
\w{4}{2}{2}{z}&=\w{2}{1}{2}{z}-\w{2}{2}{2}{z}
\end{align}
When $\alpha < 2\beta$ similar relationships exist, but require introducing terms with higher powers of $q$ than occur in the original Fourier transform. The transform does not exist when $\alpha>2\beta$. In general, we have attempted to minimize the number of $w$ functions which appear in the expressions throughout this paper rather than expand them in this way.
