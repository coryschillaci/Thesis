\chapter{Summation over spectator momentum}

\section{\label{app:momSums}Momentum space}

In this section we give explicit expressions for the sum over spectator momentum $\vec{k}_\gamma$ in the momentum space expressions of Section~\ref{sec:momentum} assuming the zero-temperature Fermi-Dirac density of states. For terms where only a single pion's momentum transfer depends on $\vec{k}_\gamma$ we have three distinct integrals of the form 
\begin{equation}
\Gamma_{\alpha}(k) = \frac{k^{2-\alpha}}{m_\pi^3}\int^{k_\gamma<k_F}\frac{d^3\vec{k}_\gamma}{(2\pi)^3}  \frac{\left\{1,\vec{k}_\gamma\cdot\hat{k},k_\gamma^2\right\}_\alpha}{(\vec{k}-\vec{k}_\gamma)^2+m_\pi^2}
\end{equation}
where $\left\{1,k_\gamma\cos\theta,k_\gamma^2\right\}_\alpha$ indicates that the numerator is $1$ for $\alpha=0$, $k_\gamma\cos\theta$ when $\alpha=1$, etc. We have suppressed the $n,p$ index on the Fermi momenta in this appendix, but remind the reader that these integrals appear in linear combinations of the isospin components.

The results of analytically integrating over the Fermi sphere are 
 \begin{align}
 \begin{split}
 \Gamma_{0}(k)&=\frac{k^2}{m_\pi^3}\int^{k_\gamma<k_F}\frac{d^3\vec{k}_\gamma}{(2\pi)^3} \frac{1}{(\vec{k}-\vec{k}_\gamma)^2+m_\pi^2} \\
 &=\frac{3}{4}\frac{\rho}{m_\pi^3}\left[  \tilde{k}^2 -\tilde{m}_\pi\tilde{k}^2\left(\arctan\frac{1-\tilde{k}}{\tilde{m}_\pi}+\arctan\frac{1+\tilde{k}}{\tilde{m}_\pi}\right)\right. \\
 &\left.\qquad\qquad\qquad\qquad\qquad+\frac{1}{4}\left(1-\tilde{k}^2+\tilde{m}_\pi^2\right)\log\frac{\tilde{m}_\pi^2+(\tilde{k}+1)^2}{\tilde{m}_\pi^2+(\tilde{k}-1)^2} \right],
 \end{split} \\
 %
  \begin{split}
 \Gamma_{1}(k)&=\frac{k}{m_\pi^3}\int^{k_\gamma<k_F}\frac{d^3\vec{k}_\gamma}{(2\pi)^3} \frac{\vec{k}_\gamma\cdot\hat{k}}{(\vec{k}-\vec{k}_\gamma)^2+m_\pi^2} \\
 &=\frac{3}{4}\frac{\rho}{m_\pi^3}\left[  (3\tilde{k}^2-\tilde{m}_\pi^2-1)-4\tilde{k}^2\tilde{m}_\pi\left(\arctan\frac{1-\tilde{k}}{\tilde{m}_\pi}+\arctan\frac{1+\tilde{k}}{\tilde{m}_\pi}\right)\right. \\
 &\left.\qquad\qquad\qquad\qquad\qquad+\frac{(1+\tilde{m}_\pi^2)^2-3\tilde{k}^4+2\tilde{k}^2(1+3\tilde{m}_\pi^2)}{4\tilde{k}}\log\frac{\tilde{m}_\pi^2+(\tilde{k}+1)^2}{\tilde{m}_\pi^2+(\tilde{k}-1)^2} \right],
 \end{split} \\
 %
  \begin{split}
 \Gamma_{2}(k)&=\frac{k}{m_\pi^3}\int^{k_\gamma<k_F}\frac{d^3\vec{k}_\gamma}{(2\pi)^3} \frac{k_\gamma^2}{(\vec{k}-\vec{k}_\gamma)^2+m_\pi^2} \\
 &=\frac{1}{2}\frac{\rho}{m_\pi^3}\left[  (3\tilde{k}^2-9\tilde{m}_\pi^2-1)-6\tilde{m}_\pi(\tilde{k}^2-\tilde{m}_\pi^2)\left(\arctan\frac{1-\tilde{k}}{\tilde{m}_\pi}+\arctan\frac{1+\tilde{k}}{\tilde{m}_\pi}\right)\right. \\
 &\left.\qquad\qquad\qquad\qquad\qquad+\frac{3\left(\tilde{k}^4+\tilde{m}_\pi^4-6\tilde{k}^2\tilde{m}_\pi^2-1\right)}{4\tilde{k}}\log\frac{\tilde{m}_\pi^2+(\tilde{k}+1)^2}{\tilde{m}_\pi^2+(\tilde{k}-1)^2} \right].
 \end{split} 
  \end{align}
All variables appear in the dimensionless combinations $\tilde{m}_\pi=m_\pi/k_F^{n,p}$ and $\tilde{k}=k/k_F^{n,p}$. For actual calculations, one can replace $k_F$ with the observable densities using \eqref{eq:kF}.

\section{\label{app:rhohat}Coordinate space}

Dealing with the Fermi gas momentum is much simpler in coordinate space. After Fourier transforming the summed diagrams back into a configuration space potential, in all cases the dependence on the spectator momentum appears through the integral
\begin{equation}
\hat{\rho}(|\rot-\rotp|) = \frac{1}{m_\pi^3}\int \frac{d^3k_\gamma}{(2\pi)^3}n(k_\gamma)e^{i\vec{k}_\gamma \cdot(\rot-\rotp)}
\end{equation}
Using our standard choice $n(k_\gamma)=\theta(k_\gamma-k_F)$ this integral can be performed analytically,
\begin{equation}\begin{split}
\hat{\rho}(r) &= \frac{1}{(2\pi)^2}\frac{1}{m_\pi^3}\int_{-1}^{-1}du \int_0^{k_F} k_\gamma^2dk_\gamma \;e^{i \vec{k}_\gamma r u} \\
&= \frac{1}{(2\pi)^2}\frac{1}{m_\pi^3} \int_0^{k_F} k_\gamma^2 dk_\gamma \;\frac{e^{i\vec{k}_\gamma r}-e^{-i\vec{k}_\gamma r}}{ik_\gamma r} \\
&= \frac{1}{(2\pi)^2}\frac{1}{m_\pi^3} \int_0^{k_F} k_\gamma^2 dk_\gamma \;\frac{2 \sin(k_\gamma r)}{k_\gamma r} \\
&= \frac{k_F^2}{2\pi^2}\frac{1}{m_\pi^3} \frac{j_1(k_\gamma r)}{r}.
\end{split}
\end{equation}
We can replace the Fermi momentum with the density using \eqref{eq:kF} to obtain
\begin{equation}
\hat{\rho}(r) = \frac{1}{2}\left(\frac{3\rho}{\pi^2 m_\pi^3}\right)^{2/3} \frac{j_1( [3\pi^2\rho]^{1/3} r)}{m_\pi r}.
\end{equation}
Note that the factor of $1/2$ is not included in \eqref{eq:hatdensities} which means that
\begin{equation}
\int \frac{d^3k_\gamma}{(2\pi)^3}\big[ n(k_\gamma)\pm n(k_\gamma)\big]e^{i\vec{k}_\gamma \cdot(\rot-\rotp)} =m_\pi^3 \frac{ \rhohat{0,1}{|\rot-\rotp)|}}{2}.
\end{equation}

\chapter{\label{app:wNotation} Fourier integrals}

In an attempt at organizing the terms in the coordinate space effective potential, we introduce the following notation for the dimensionless radial functions $\w{\alpha}{\beta}{\ell}{ m r }$:

\begin{align}
i^\ell \, Y_\ell(\hat{r}) \, \w{\alpha}{\beta}{\ell}{ m r}  &= \frac{4 \pi}{m^{3+\alpha-2\beta} }  \int \frac{d^3 q}{(2\pi)^3} \frac{q^\alpha}{(q^2+m^2)^\beta} Y_\ell(\hat{q}) \, e^{i \mathbf{q} \cdot \mathbf{r}  },
\end{align}
which implies that
\begin{align}\label{eq:wDef}
\w{\alpha}{\beta}{\ell}{z} =  \frac{2}{\pi} \int dk \, k^2 \, \frac{k^\alpha}{(1+k^2)^\beta} j_\ell(k z)
\end{align}
with $\vec{z} \equiv m \vec{r}$ and $k$ both dimensionless. In Table~\ref{table:wTable} we give the integrated results for selected values of $\alpha, \beta, \ell$. The first function we recover, for the case of $(\alpha, \beta, \ell ) = (0,1,0)$, is nothing but a Yukawa potential.

\begin{table}
\begin{center}
\begin{tabular}{| c c c | c |}
\hline
$\alpha$ & $\beta$ & $\ell$ & $ \w{\alpha}{\beta}{\ell}{z}$ \\
\hline
0 & 0 & 0 & $ 4\pi \delta^{(3)}(\vec{z}) $ \\
0 & 1 & 0 & $ \frac{\displaystyle e^{-z}}{\displaystyle z} $ \\
2 & 1 & 0 & $ -\frac{\displaystyle e^{-z}}{\displaystyle z} + 4\pi \delta^{(3)}(\vec{z}) $ \\
1 & 1 & 1 &  $\frac{\displaystyle e^{-z}}{\displaystyle z} (1 + \tfrac{1}{z})  $ \\
2 & 1 & 2 & $ \frac{\displaystyle e^{-z}}{\displaystyle z} (1+ \tfrac{3}{z} + \tfrac{3}{z^2} )$ \\
0 & 2 & 0 & $ \frac{\displaystyle e^{-z}}{2} $ \\
2 & 2 & 0 & $ \frac{\displaystyle e^{-z}}{\displaystyle z} (-\tfrac{z}{2} + 1 ) $ \\
2 & 2 & 2 & $  \frac{\displaystyle e^{-z}}{\displaystyle z} (\tfrac{z}{2} + \tfrac{1}{2} ) $ \\
4 & 2 & 0 & $  \frac{\displaystyle e^{-z}}{\displaystyle z} (\tfrac{z}{2} - 2 ) + 4 \pi \delta^3( \vec{z} )  $ \\
4 & 2 & 2 & $  \frac{\displaystyle e^{-z}}{\displaystyle z} ( - \tfrac{z}{2} + \tfrac{1}{2} + \tfrac{3}{z} + \tfrac{3}{z^2} ) $ \\
\hline
\end{tabular}
\end{center}
\caption{\label{table:wTable} Table of selected values for $\w{\alpha}{\beta}{\ell}{z}$ found by integrating equation \eqref{eq:wDef}.}
\end{table}

To further demonstrate the use of this integral, we now derive the familiar tensor force resulting from one pion exchange. This amplitude is related to the integral
\begin{equation} 
V_\pi( \mathbf{r}) =  -\frac{g_A^2 }{4 F_\pi^2} \:\taudot \int \frac{d^3 q } {(2 \pi)^3 } \frac{(\sigma^{(1)} \cdot \mathbf{q} ) (\sigma^{(2)} \cdot \mathbf{q} ) } {q^2+m_\pi^2} e^{i \mathbf{q} \cdot \mathbf{r} }. 
\end{equation}
We can rewrite the numerator in the integral as
\begin{equation} (\sigma^{(1)} \cdot \mathbf{q}) ( \sigma^{(2)} \cdot \mathbf{q} )= q^2 \left(\sqrt{\frac{8\pi}{15}} [\sigma^{(1)} \otimes \sigma^{(2)}]_2 \cdot Y_2(\hat{q}) + \tfrac{1}{3} \sigma^{(1)} \cdot \sigma^{(2)}  Y_0(\hat{q}) \right),\end{equation}
which then gives
\begin{equation} 
V_\pi( \mathbf{r}) = - \frac{g_A^2 }{4 F_\pi^2} \taudot \left(\frac{m_\pi^3}{4 \pi}\right) \left\{ - \sqrt{\frac{8\pi}{15}} [\sigma_1 \otimes \sigma_2]_2 \cdot Y_2(\hat{r}) \w{2}{1}{2}{m_\pi r} + \frac{1}{3} \sigma_1 \cdot \sigma_2 \w{2}{1}{0}{m_\pi r } \right\},  \end{equation}
or, plugging in explicit values for the $w$'s from Table \ref{table:wTable} and recalling that the tensor operator $S_{12}$ is conventionally defined
\begin{equation}
S_{12} \equiv   3\vec{\sigma}_1\cdot \hat{r}\vec{\sigma}_2\cdot \hat{r} -\sigmadot = 3\:\sqrt{\frac{8\pi}{15}} \:[\sigma_1 \otimes \sigma_2]_2 \cdot Y_2(\hat{r}) , 
\end{equation}
we have
\begin{equation} V_\pi( \mathbf{r})= \frac{g_A^2 }{4 F_\pi^2}  \left(\frac{m_\pi^3}{12 \pi}\right) \left\{ \frac{e^{-m_\pi r}}{m_\pi r} \left[ S_{12}\left(1+\frac{3}{m_\pi r}+ \frac{3}{m_\pi^2 r^2} \right) +  \sigma_1 \cdot \sigma_2 \right] - 4\pi  \sigma_1 \cdot \sigma_2 \delta^3(m_\pi \mathbf{r}) \right\}, 
\end{equation}
which is the familiar result.

Representations in terms of these functions are not unique. When $\alpha=2\beta$, one may rewrite the numerator in \eqref{eq:wDef} using only powers of $q$ less than $\alpha$. As a concrete example, consider the case for $\w{2}{1}{0}{z}$. Because
\begin{equation}
\frac{q^2}{q^2+m^2}=1-\frac{m^2}{q^2+m^2}
\end{equation}
we immediately see that $\w{2}{1}{0}{z}=\w{0}{0}{0}{z}-\w{0}{1}{0}{z}$, a result which can be confirmed by inspection from Table~\ref{table:wTable}. This rewriting also makes clear the origin of the short-range delta function terms in the Fourier transforms of the form $\w{2\beta}{\beta}{0}{z}$. Other relationships include,
\begin{align}
\w{4}{2}{0}{z}&=\w{0}{0}{0}{z}-2 \w{2}{2}{0}{z}-\w{0}{2}{0}{z} \\
\w{4}{2}{2}{z}&=\w{2}{1}{2}{z}-\w{2}{2}{2}{z}
\end{align}
When $\alpha < 2\beta$ similar relationships exist, but require introducing terms with higher powers of $q$ than occur in the original Fourier transform. The transform does not exist when $\alpha>2\beta$. In general, we have attempted to minimize the number of $w$ functions which appear in the expressions throughout this paper rather than expand them in this way.
