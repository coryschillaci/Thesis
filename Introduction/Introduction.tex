\chapter{\label{chap:Introduction}Introduction: Effective Theories}

The concept of an effective theory is ubiquitous in physics. Fundamentally, the idea is to solve an intractable problem by making certain approximations which simplify the problem while still retaining predictive power. Although simplifying approximations have a long history in physics,  the work of Wilson \cite{Wilson197475} in the 1970s paved the way for a more rigorous modern perspective. From a contemporary viewpoint, effective theories are useful when the details of physics above some energy scale $\Lambda$ (or below some length scale) are unknown or very complicated. For example, it is believed that the standard model of particle physics is an effective theory. Infinities, which are removed by renormalization, arise because there is some energy scale at which new physics comes into play and the model is therefore not to be taken seriously at arbitrarily high energies when unknown physics contributes significantly. Furthermore, not all of the standard model particles and interactions are relevant for all physical phenomena. We can therefore formally generate new effective theories systematically from the standard model by integrating out degrees of freedom such as very massive particles.

Weinberg pioneered another approach to generating an effective field theory describing interactions of hadrons \cite{WEINBERG1990288}. In his framework, one should consider all possible terms which respect the approximate chiral symmetry of quantum chromodynamics (QCD) at low energies. Since there are an infinite number of such terms, a power counting scheme is necessary to group interactions into a finite numbers of terms at each order in a low-momentum expansion\footnote{The relevant expansion parameter is $Q/\Lambda_{\text{QCD}}$ where $Q$ represents some small momentum scale and the chiral symmetry breaking scale $\Lambda_{\text{QCD}}\approx 1\text{GeV}$.}. The expansion contains a number of parameters, called low energy constants (LECs), multiplying the various terms. These are constrained by the chiral symmetry, and their values are determined by fitting calculated observables to their experimental values. When performing calculations, one should also take the energy cutoff $\Lambda_{\text{QCD}}$ seriously as an ultraviolet cutoff (we have no reason to extend our model beyond this scale) \cite{Epelbaum2013}. 

Weinberg's approach, referred to as chiral perturbation theory (ChPT), has enjoyed significant popularity in the nuclear physics community. Derivation of a non-relativistic potential describing the interactions of nucleons is a primary goal of the field, and the success of chiral potentials combined with their rigorous basis on underlying symmetries helped them to largely supplant the phenomenological potentials previously in use. For a thorough introduction to the subject, see \cite{Machleidt20111}.

One generic feature of effective theories for composite particles is the generation of interactions between three or more particles. Chapters \ref{chap:3to2} and \ref{chap:localExpansion} describe a procedure for generating two-body density-dependent potentials which approximate the three-body interactions of the chiral potential at next-to-next-to-leading order (N$^2$LO).

Effective theories are also useful in completely non-relativistic formulations of quantum physics, although this is often less appreciated than for their applications in field theories. The primary features of an effective theory which we want to preserve from the rigorous field-theoretic perspective are \cite{Lepage:1997}:
\begin{itemize}
\item Include the correct low energy (long-range) physics explicitly. 
\item Identify a cutoff energy for the theory. This should lie between the low energy physics, which are known, and the unknown or intractable high energy physics. 
\item Incorporate the unknown high energy (short-range) physics by adding additional, model-independent terms and fitting the cutoff-dependent coefficients to physical observables.
\end{itemize}

From one viewpoint, a hard momentum cutoff is equivalent to reducing the basis states in momentum space used in calculations. We can thus equivalently think of restricting our calculations to a finite subspace in the full Hilbert space. This point of view is often taken for practical reasons in numerical calculations. The only other basis in which center of mass and relative coordinates cleanly separate for the N-body problem is that of harmonic oscillator (HO) eigenstates, and naturally we can also consider imposing the energy cutoff by including only HO shells with energy less than $\Lambda$. 

A detailed description of harmonic-oscillator-based effective theory (HOBET) is given in \cite{PhysRevC.77.034005}. A particular issue with using a finite basis of Harmonic oscillator states to regularize is that the states do not accurately represent very low momentum states (it is an expansion around $k\sim 1/b$). Additional corrections are therefore needed to properly treat the long-range part of the potential exactly, and these introduce an energy dependence to the effective Hamiltonian in the finite basis.

In Chapter \ref{chap:SOC}, we consider a special case of the harmonic-oscillator-based effective theory. For cold atom systems, the confining potential is often approximately harmonic. The machinery of the HOBET greatly simplifies in a harmonic confining potential, as the there is no need to perform the long-range corrections. We therefore use the effective theory to non-perturbatively explore the two-body spectra of spin-$1/2$ fermions in isotropic harmonic traps with external spin-orbit potentials. Interatomic potentials are high-energy compared to the very low temperatures of the atomic gasses, and are absorbed into short range two-body interactions. Results are presented for experimentally realistic forms of the spin-orbit coupling: a pure Rashba coupling, Rashba and Dresselhaus couplings in equal parts, and a Weyl-type coupling. The technique is easily adapted to bosonic systems and other forms of spin-orbit coupling.